\documentclass{article}
\usepackage{fullpage}

%load needed packages
\usepackage{graphicx}
\usepackage{array}
\usepackage{booktabs}
\usepackage[utf8]{inputenc}
\usepackage[T1]{fontenc}
\usepackage{hyperref}


\usepackage{float}  % Necesario para [H]
\usepackage{listings}
\usepackage{xcolor}

\definecolor{codegreen}{HTML}{5AB2FF}
\definecolor{morado}{HTML}{AD88C6}
\definecolor{BG}{HTML}{EEEEEE}
\definecolor{azul}{HTML}{4D869C}
\definecolor{sqlblue}{HTML}{FF8C00} % Color para las palabras clave SQL
\usepackage{listings}
\usepackage{xcolor}


%estilo python
\usepackage{xcolor}

% Define the colors for the style
\definecolor{BG}{rgb}{0.95,0.95,0.95}  % Background color
\definecolor{keywordcolor}{rgb}{0.0,0.0,1.0} % Blue for keywords
\definecolor{commentcolor}{rgb}{0.0,0.5,0.0} % Green for comments
\definecolor{stringcolor}{rgb}{1.0,0.0,0.0}  % Red for strings
\definecolor{attributecolor}{rgb}{0.8,0.3,0.8} % Purple for attributes
\definecolor{importcolor}{rgb}{0.0,0.6,0.6} % Teal for import statements

% Define the style for Python code
\lstdefinestyle{mypython}{
	backgroundcolor=\color{BG},   % Background color
	basicstyle=\footnotesize\ttfamily,  
	breaklines=true,                  
	language=Python,                  
	keywordstyle=\color{keywordcolor},    
	commentstyle=\color{commentcolor}, 
	stringstyle=\color{stringcolor},
	frame=shadowbox, 
	morekeywords={model},  % Add 'model' to keywords
	keywordstyle=[2]\color{importcolor}, % Color for import statements
	sensitive=true,       % Case sensitive
	morecomment=[s]{"""}{"} % Allows for multi-line strings
}



\lstset{style=mypython}
% Estilo para DDL
\lstdefinestyle{ddlstyle}{
	language=SQL,
	backgroundcolor=\color{BG},
	commentstyle=\color{codegreen},
	basicstyle=\ttfamily\small,
	keywordstyle=\color{azul},
	stringstyle=\color{morado},
	showstringspaces=false,
	breaklines=true,
	frame=shadowbox,
	numbers=left,
	numberstyle=\tiny\color{gray},
	captionpos=b,
}

% Estilo para SQL
\lstdefinestyle{sqlstyle}{
	language=SQL,
	backgroundcolor=\color{BG},
	commentstyle=\color{codegreen},
	basicstyle=\ttfamily\small,
	keywordstyle=\color{sqlblue}, % Color diferente para palabras clave SQL
	stringstyle=\color{morado},
	showstringspaces=false,
	breaklines=true,
	frame=shadowbox,
	numbers=left,
	numberstyle=\tiny\color{gray},
	captionpos=b,
}

\begin{document}



% Portada
\begin{titlepage}
	\centering
	\vspace*{3cm}
	
	% Título destacado
	{\Huge \textbf{Lab 1: Clustering}\\[0.5cm]}
	
	% Espacio y logotipo (si lo tienes, por ejemplo el logo de tu universidad)
	\vspace{2cm}
	\includegraphics[width=0.3\textwidth]{images/uma_logo.jpg}\\[1cm]
	
	% Nombre del autor
	{\LARGE \textbf{Alejandro Silva Rodríguez}\\[0.5cm]}
	{\LARGE \textbf{Marta Cuevas Rodríguez}\\[0.5cm]}
	{\large \textit{Aprendizaje Computacional}\\
		Universidad de Málaga\\
		}
	
	\vfill
	
	% Fecha en la parte inferior de la página
	{\large Septiembre 2024}
\end{titlepage}

% indice
\tableofcontents

\newpage

\section{Introduction}

\section{Objectives}

\section{Methodology and Results}

\section{Conclusion}


\section{Repository Access}

All additional information, including the source code and full documentation of this project, is available in the GitHub repository \cite{cuevas2024github}.


% Incluir la bibliografía
\bibliographystyle{plain}  % Estilo de la bibliografía (por ejemplo, plain, alpha, ieee, etc.)
\bibliography{bibli}  % Nombre del archivo .bib sin la extensión

\end{document}
